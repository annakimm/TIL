
% Default to the notebook output style

    


% Inherit from the specified cell style.




    
\documentclass[11pt]{article}

    
    
    \usepackage[T1]{fontenc}
    % Nicer default font (+ math font) than Computer Modern for most use cases
    \usepackage{mathpazo}

    % Basic figure setup, for now with no caption control since it's done
    % automatically by Pandoc (which extracts ![](path) syntax from Markdown).
    \usepackage{graphicx}
    % We will generate all images so they have a width \maxwidth. This means
    % that they will get their normal width if they fit onto the page, but
    % are scaled down if they would overflow the margins.
    \makeatletter
    \def\maxwidth{\ifdim\Gin@nat@width>\linewidth\linewidth
    \else\Gin@nat@width\fi}
    \makeatother
    \let\Oldincludegraphics\includegraphics
    % Set max figure width to be 80% of text width, for now hardcoded.
    \renewcommand{\includegraphics}[1]{\Oldincludegraphics[width=.8\maxwidth]{#1}}
    % Ensure that by default, figures have no caption (until we provide a
    % proper Figure object with a Caption API and a way to capture that
    % in the conversion process - todo).
    \usepackage{caption}
    \DeclareCaptionLabelFormat{nolabel}{}
    \captionsetup{labelformat=nolabel}

    \usepackage{adjustbox} % Used to constrain images to a maximum size 
    \usepackage{xcolor} % Allow colors to be defined
    \usepackage{enumerate} % Needed for markdown enumerations to work
    \usepackage{geometry} % Used to adjust the document margins
    \usepackage{amsmath} % Equations
    \usepackage{amssymb} % Equations
    \usepackage{textcomp} % defines textquotesingle
    % Hack from http://tex.stackexchange.com/a/47451/13684:
    \AtBeginDocument{%
        \def\PYZsq{\textquotesingle}% Upright quotes in Pygmentized code
    }
    \usepackage{upquote} % Upright quotes for verbatim code
    \usepackage{eurosym} % defines \euro
    \usepackage[mathletters]{ucs} % Extended unicode (utf-8) support
    \usepackage[utf8x]{inputenc} % Allow utf-8 characters in the tex document
    \usepackage{fancyvrb} % verbatim replacement that allows latex
    \usepackage{grffile} % extends the file name processing of package graphics 
                         % to support a larger range 
    % The hyperref package gives us a pdf with properly built
    % internal navigation ('pdf bookmarks' for the table of contents,
    % internal cross-reference links, web links for URLs, etc.)
    \usepackage{hyperref}
    \usepackage{longtable} % longtable support required by pandoc >1.10
    \usepackage{booktabs}  % table support for pandoc > 1.12.2
    \usepackage[inline]{enumitem} % IRkernel/repr support (it uses the enumerate* environment)
    \usepackage[normalem]{ulem} % ulem is needed to support strikethroughs (\sout)
                                % normalem makes italics be italics, not underlines
    

    
    
    % Colors for the hyperref package
    \definecolor{urlcolor}{rgb}{0,.145,.698}
    \definecolor{linkcolor}{rgb}{.71,0.21,0.01}
    \definecolor{citecolor}{rgb}{.12,.54,.11}

    % ANSI colors
    \definecolor{ansi-black}{HTML}{3E424D}
    \definecolor{ansi-black-intense}{HTML}{282C36}
    \definecolor{ansi-red}{HTML}{E75C58}
    \definecolor{ansi-red-intense}{HTML}{B22B31}
    \definecolor{ansi-green}{HTML}{00A250}
    \definecolor{ansi-green-intense}{HTML}{007427}
    \definecolor{ansi-yellow}{HTML}{DDB62B}
    \definecolor{ansi-yellow-intense}{HTML}{B27D12}
    \definecolor{ansi-blue}{HTML}{208FFB}
    \definecolor{ansi-blue-intense}{HTML}{0065CA}
    \definecolor{ansi-magenta}{HTML}{D160C4}
    \definecolor{ansi-magenta-intense}{HTML}{A03196}
    \definecolor{ansi-cyan}{HTML}{60C6C8}
    \definecolor{ansi-cyan-intense}{HTML}{258F8F}
    \definecolor{ansi-white}{HTML}{C5C1B4}
    \definecolor{ansi-white-intense}{HTML}{A1A6B2}

    % commands and environments needed by pandoc snippets
    % extracted from the output of `pandoc -s`
    \providecommand{\tightlist}{%
      \setlength{\itemsep}{0pt}\setlength{\parskip}{0pt}}
    \DefineVerbatimEnvironment{Highlighting}{Verbatim}{commandchars=\\\{\}}
    % Add ',fontsize=\small' for more characters per line
    \newenvironment{Shaded}{}{}
    \newcommand{\KeywordTok}[1]{\textcolor[rgb]{0.00,0.44,0.13}{\textbf{{#1}}}}
    \newcommand{\DataTypeTok}[1]{\textcolor[rgb]{0.56,0.13,0.00}{{#1}}}
    \newcommand{\DecValTok}[1]{\textcolor[rgb]{0.25,0.63,0.44}{{#1}}}
    \newcommand{\BaseNTok}[1]{\textcolor[rgb]{0.25,0.63,0.44}{{#1}}}
    \newcommand{\FloatTok}[1]{\textcolor[rgb]{0.25,0.63,0.44}{{#1}}}
    \newcommand{\CharTok}[1]{\textcolor[rgb]{0.25,0.44,0.63}{{#1}}}
    \newcommand{\StringTok}[1]{\textcolor[rgb]{0.25,0.44,0.63}{{#1}}}
    \newcommand{\CommentTok}[1]{\textcolor[rgb]{0.38,0.63,0.69}{\textit{{#1}}}}
    \newcommand{\OtherTok}[1]{\textcolor[rgb]{0.00,0.44,0.13}{{#1}}}
    \newcommand{\AlertTok}[1]{\textcolor[rgb]{1.00,0.00,0.00}{\textbf{{#1}}}}
    \newcommand{\FunctionTok}[1]{\textcolor[rgb]{0.02,0.16,0.49}{{#1}}}
    \newcommand{\RegionMarkerTok}[1]{{#1}}
    \newcommand{\ErrorTok}[1]{\textcolor[rgb]{1.00,0.00,0.00}{\textbf{{#1}}}}
    \newcommand{\NormalTok}[1]{{#1}}
    
    % Additional commands for more recent versions of Pandoc
    \newcommand{\ConstantTok}[1]{\textcolor[rgb]{0.53,0.00,0.00}{{#1}}}
    \newcommand{\SpecialCharTok}[1]{\textcolor[rgb]{0.25,0.44,0.63}{{#1}}}
    \newcommand{\VerbatimStringTok}[1]{\textcolor[rgb]{0.25,0.44,0.63}{{#1}}}
    \newcommand{\SpecialStringTok}[1]{\textcolor[rgb]{0.73,0.40,0.53}{{#1}}}
    \newcommand{\ImportTok}[1]{{#1}}
    \newcommand{\DocumentationTok}[1]{\textcolor[rgb]{0.73,0.13,0.13}{\textit{{#1}}}}
    \newcommand{\AnnotationTok}[1]{\textcolor[rgb]{0.38,0.63,0.69}{\textbf{\textit{{#1}}}}}
    \newcommand{\CommentVarTok}[1]{\textcolor[rgb]{0.38,0.63,0.69}{\textbf{\textit{{#1}}}}}
    \newcommand{\VariableTok}[1]{\textcolor[rgb]{0.10,0.09,0.49}{{#1}}}
    \newcommand{\ControlFlowTok}[1]{\textcolor[rgb]{0.00,0.44,0.13}{\textbf{{#1}}}}
    \newcommand{\OperatorTok}[1]{\textcolor[rgb]{0.40,0.40,0.40}{{#1}}}
    \newcommand{\BuiltInTok}[1]{{#1}}
    \newcommand{\ExtensionTok}[1]{{#1}}
    \newcommand{\PreprocessorTok}[1]{\textcolor[rgb]{0.74,0.48,0.00}{{#1}}}
    \newcommand{\AttributeTok}[1]{\textcolor[rgb]{0.49,0.56,0.16}{{#1}}}
    \newcommand{\InformationTok}[1]{\textcolor[rgb]{0.38,0.63,0.69}{\textbf{\textit{{#1}}}}}
    \newcommand{\WarningTok}[1]{\textcolor[rgb]{0.38,0.63,0.69}{\textbf{\textit{{#1}}}}}
    
    
    % Define a nice break command that doesn't care if a line doesn't already
    % exist.
    \def\br{\hspace*{\fill} \\* }
    % Math Jax compatability definitions
    \def\gt{>}
    \def\lt{<}
    % Document parameters
    \title{05\_function\_summary}
    
    
    

    % Pygments definitions
    
\makeatletter
\def\PY@reset{\let\PY@it=\relax \let\PY@bf=\relax%
    \let\PY@ul=\relax \let\PY@tc=\relax%
    \let\PY@bc=\relax \let\PY@ff=\relax}
\def\PY@tok#1{\csname PY@tok@#1\endcsname}
\def\PY@toks#1+{\ifx\relax#1\empty\else%
    \PY@tok{#1}\expandafter\PY@toks\fi}
\def\PY@do#1{\PY@bc{\PY@tc{\PY@ul{%
    \PY@it{\PY@bf{\PY@ff{#1}}}}}}}
\def\PY#1#2{\PY@reset\PY@toks#1+\relax+\PY@do{#2}}

\expandafter\def\csname PY@tok@w\endcsname{\def\PY@tc##1{\textcolor[rgb]{0.73,0.73,0.73}{##1}}}
\expandafter\def\csname PY@tok@c\endcsname{\let\PY@it=\textit\def\PY@tc##1{\textcolor[rgb]{0.25,0.50,0.50}{##1}}}
\expandafter\def\csname PY@tok@cp\endcsname{\def\PY@tc##1{\textcolor[rgb]{0.74,0.48,0.00}{##1}}}
\expandafter\def\csname PY@tok@k\endcsname{\let\PY@bf=\textbf\def\PY@tc##1{\textcolor[rgb]{0.00,0.50,0.00}{##1}}}
\expandafter\def\csname PY@tok@kp\endcsname{\def\PY@tc##1{\textcolor[rgb]{0.00,0.50,0.00}{##1}}}
\expandafter\def\csname PY@tok@kt\endcsname{\def\PY@tc##1{\textcolor[rgb]{0.69,0.00,0.25}{##1}}}
\expandafter\def\csname PY@tok@o\endcsname{\def\PY@tc##1{\textcolor[rgb]{0.40,0.40,0.40}{##1}}}
\expandafter\def\csname PY@tok@ow\endcsname{\let\PY@bf=\textbf\def\PY@tc##1{\textcolor[rgb]{0.67,0.13,1.00}{##1}}}
\expandafter\def\csname PY@tok@nb\endcsname{\def\PY@tc##1{\textcolor[rgb]{0.00,0.50,0.00}{##1}}}
\expandafter\def\csname PY@tok@nf\endcsname{\def\PY@tc##1{\textcolor[rgb]{0.00,0.00,1.00}{##1}}}
\expandafter\def\csname PY@tok@nc\endcsname{\let\PY@bf=\textbf\def\PY@tc##1{\textcolor[rgb]{0.00,0.00,1.00}{##1}}}
\expandafter\def\csname PY@tok@nn\endcsname{\let\PY@bf=\textbf\def\PY@tc##1{\textcolor[rgb]{0.00,0.00,1.00}{##1}}}
\expandafter\def\csname PY@tok@ne\endcsname{\let\PY@bf=\textbf\def\PY@tc##1{\textcolor[rgb]{0.82,0.25,0.23}{##1}}}
\expandafter\def\csname PY@tok@nv\endcsname{\def\PY@tc##1{\textcolor[rgb]{0.10,0.09,0.49}{##1}}}
\expandafter\def\csname PY@tok@no\endcsname{\def\PY@tc##1{\textcolor[rgb]{0.53,0.00,0.00}{##1}}}
\expandafter\def\csname PY@tok@nl\endcsname{\def\PY@tc##1{\textcolor[rgb]{0.63,0.63,0.00}{##1}}}
\expandafter\def\csname PY@tok@ni\endcsname{\let\PY@bf=\textbf\def\PY@tc##1{\textcolor[rgb]{0.60,0.60,0.60}{##1}}}
\expandafter\def\csname PY@tok@na\endcsname{\def\PY@tc##1{\textcolor[rgb]{0.49,0.56,0.16}{##1}}}
\expandafter\def\csname PY@tok@nt\endcsname{\let\PY@bf=\textbf\def\PY@tc##1{\textcolor[rgb]{0.00,0.50,0.00}{##1}}}
\expandafter\def\csname PY@tok@nd\endcsname{\def\PY@tc##1{\textcolor[rgb]{0.67,0.13,1.00}{##1}}}
\expandafter\def\csname PY@tok@s\endcsname{\def\PY@tc##1{\textcolor[rgb]{0.73,0.13,0.13}{##1}}}
\expandafter\def\csname PY@tok@sd\endcsname{\let\PY@it=\textit\def\PY@tc##1{\textcolor[rgb]{0.73,0.13,0.13}{##1}}}
\expandafter\def\csname PY@tok@si\endcsname{\let\PY@bf=\textbf\def\PY@tc##1{\textcolor[rgb]{0.73,0.40,0.53}{##1}}}
\expandafter\def\csname PY@tok@se\endcsname{\let\PY@bf=\textbf\def\PY@tc##1{\textcolor[rgb]{0.73,0.40,0.13}{##1}}}
\expandafter\def\csname PY@tok@sr\endcsname{\def\PY@tc##1{\textcolor[rgb]{0.73,0.40,0.53}{##1}}}
\expandafter\def\csname PY@tok@ss\endcsname{\def\PY@tc##1{\textcolor[rgb]{0.10,0.09,0.49}{##1}}}
\expandafter\def\csname PY@tok@sx\endcsname{\def\PY@tc##1{\textcolor[rgb]{0.00,0.50,0.00}{##1}}}
\expandafter\def\csname PY@tok@m\endcsname{\def\PY@tc##1{\textcolor[rgb]{0.40,0.40,0.40}{##1}}}
\expandafter\def\csname PY@tok@gh\endcsname{\let\PY@bf=\textbf\def\PY@tc##1{\textcolor[rgb]{0.00,0.00,0.50}{##1}}}
\expandafter\def\csname PY@tok@gu\endcsname{\let\PY@bf=\textbf\def\PY@tc##1{\textcolor[rgb]{0.50,0.00,0.50}{##1}}}
\expandafter\def\csname PY@tok@gd\endcsname{\def\PY@tc##1{\textcolor[rgb]{0.63,0.00,0.00}{##1}}}
\expandafter\def\csname PY@tok@gi\endcsname{\def\PY@tc##1{\textcolor[rgb]{0.00,0.63,0.00}{##1}}}
\expandafter\def\csname PY@tok@gr\endcsname{\def\PY@tc##1{\textcolor[rgb]{1.00,0.00,0.00}{##1}}}
\expandafter\def\csname PY@tok@ge\endcsname{\let\PY@it=\textit}
\expandafter\def\csname PY@tok@gs\endcsname{\let\PY@bf=\textbf}
\expandafter\def\csname PY@tok@gp\endcsname{\let\PY@bf=\textbf\def\PY@tc##1{\textcolor[rgb]{0.00,0.00,0.50}{##1}}}
\expandafter\def\csname PY@tok@go\endcsname{\def\PY@tc##1{\textcolor[rgb]{0.53,0.53,0.53}{##1}}}
\expandafter\def\csname PY@tok@gt\endcsname{\def\PY@tc##1{\textcolor[rgb]{0.00,0.27,0.87}{##1}}}
\expandafter\def\csname PY@tok@err\endcsname{\def\PY@bc##1{\setlength{\fboxsep}{0pt}\fcolorbox[rgb]{1.00,0.00,0.00}{1,1,1}{\strut ##1}}}
\expandafter\def\csname PY@tok@kc\endcsname{\let\PY@bf=\textbf\def\PY@tc##1{\textcolor[rgb]{0.00,0.50,0.00}{##1}}}
\expandafter\def\csname PY@tok@kd\endcsname{\let\PY@bf=\textbf\def\PY@tc##1{\textcolor[rgb]{0.00,0.50,0.00}{##1}}}
\expandafter\def\csname PY@tok@kn\endcsname{\let\PY@bf=\textbf\def\PY@tc##1{\textcolor[rgb]{0.00,0.50,0.00}{##1}}}
\expandafter\def\csname PY@tok@kr\endcsname{\let\PY@bf=\textbf\def\PY@tc##1{\textcolor[rgb]{0.00,0.50,0.00}{##1}}}
\expandafter\def\csname PY@tok@bp\endcsname{\def\PY@tc##1{\textcolor[rgb]{0.00,0.50,0.00}{##1}}}
\expandafter\def\csname PY@tok@fm\endcsname{\def\PY@tc##1{\textcolor[rgb]{0.00,0.00,1.00}{##1}}}
\expandafter\def\csname PY@tok@vc\endcsname{\def\PY@tc##1{\textcolor[rgb]{0.10,0.09,0.49}{##1}}}
\expandafter\def\csname PY@tok@vg\endcsname{\def\PY@tc##1{\textcolor[rgb]{0.10,0.09,0.49}{##1}}}
\expandafter\def\csname PY@tok@vi\endcsname{\def\PY@tc##1{\textcolor[rgb]{0.10,0.09,0.49}{##1}}}
\expandafter\def\csname PY@tok@vm\endcsname{\def\PY@tc##1{\textcolor[rgb]{0.10,0.09,0.49}{##1}}}
\expandafter\def\csname PY@tok@sa\endcsname{\def\PY@tc##1{\textcolor[rgb]{0.73,0.13,0.13}{##1}}}
\expandafter\def\csname PY@tok@sb\endcsname{\def\PY@tc##1{\textcolor[rgb]{0.73,0.13,0.13}{##1}}}
\expandafter\def\csname PY@tok@sc\endcsname{\def\PY@tc##1{\textcolor[rgb]{0.73,0.13,0.13}{##1}}}
\expandafter\def\csname PY@tok@dl\endcsname{\def\PY@tc##1{\textcolor[rgb]{0.73,0.13,0.13}{##1}}}
\expandafter\def\csname PY@tok@s2\endcsname{\def\PY@tc##1{\textcolor[rgb]{0.73,0.13,0.13}{##1}}}
\expandafter\def\csname PY@tok@sh\endcsname{\def\PY@tc##1{\textcolor[rgb]{0.73,0.13,0.13}{##1}}}
\expandafter\def\csname PY@tok@s1\endcsname{\def\PY@tc##1{\textcolor[rgb]{0.73,0.13,0.13}{##1}}}
\expandafter\def\csname PY@tok@mb\endcsname{\def\PY@tc##1{\textcolor[rgb]{0.40,0.40,0.40}{##1}}}
\expandafter\def\csname PY@tok@mf\endcsname{\def\PY@tc##1{\textcolor[rgb]{0.40,0.40,0.40}{##1}}}
\expandafter\def\csname PY@tok@mh\endcsname{\def\PY@tc##1{\textcolor[rgb]{0.40,0.40,0.40}{##1}}}
\expandafter\def\csname PY@tok@mi\endcsname{\def\PY@tc##1{\textcolor[rgb]{0.40,0.40,0.40}{##1}}}
\expandafter\def\csname PY@tok@il\endcsname{\def\PY@tc##1{\textcolor[rgb]{0.40,0.40,0.40}{##1}}}
\expandafter\def\csname PY@tok@mo\endcsname{\def\PY@tc##1{\textcolor[rgb]{0.40,0.40,0.40}{##1}}}
\expandafter\def\csname PY@tok@ch\endcsname{\let\PY@it=\textit\def\PY@tc##1{\textcolor[rgb]{0.25,0.50,0.50}{##1}}}
\expandafter\def\csname PY@tok@cm\endcsname{\let\PY@it=\textit\def\PY@tc##1{\textcolor[rgb]{0.25,0.50,0.50}{##1}}}
\expandafter\def\csname PY@tok@cpf\endcsname{\let\PY@it=\textit\def\PY@tc##1{\textcolor[rgb]{0.25,0.50,0.50}{##1}}}
\expandafter\def\csname PY@tok@c1\endcsname{\let\PY@it=\textit\def\PY@tc##1{\textcolor[rgb]{0.25,0.50,0.50}{##1}}}
\expandafter\def\csname PY@tok@cs\endcsname{\let\PY@it=\textit\def\PY@tc##1{\textcolor[rgb]{0.25,0.50,0.50}{##1}}}

\def\PYZbs{\char`\\}
\def\PYZus{\char`\_}
\def\PYZob{\char`\{}
\def\PYZcb{\char`\}}
\def\PYZca{\char`\^}
\def\PYZam{\char`\&}
\def\PYZlt{\char`\<}
\def\PYZgt{\char`\>}
\def\PYZsh{\char`\#}
\def\PYZpc{\char`\%}
\def\PYZdl{\char`\$}
\def\PYZhy{\char`\-}
\def\PYZsq{\char`\'}
\def\PYZdq{\char`\"}
\def\PYZti{\char`\~}
% for compatibility with earlier versions
\def\PYZat{@}
\def\PYZlb{[}
\def\PYZrb{]}
\makeatother


    % Exact colors from NB
    \definecolor{incolor}{rgb}{0.0, 0.0, 0.5}
    \definecolor{outcolor}{rgb}{0.545, 0.0, 0.0}



    
    % Prevent overflowing lines due to hard-to-break entities
    \sloppy 
    % Setup hyperref package
    \hypersetup{
      breaklinks=true,  % so long urls are correctly broken across lines
      colorlinks=true,
      urlcolor=urlcolor,
      linkcolor=linkcolor,
      citecolor=citecolor,
      }
    % Slightly bigger margins than the latex defaults
    
    \geometry{verbose,tmargin=1in,bmargin=1in,lmargin=1in,rmargin=1in}
    
    

    \begin{document}
    
    
    \maketitle
    
    

    
    \paragraph{Function}\label{function}

\begin{itemize}
\item
  def로 선언, 함수가 호출되면 함수 안에 있는 코드 실행
\item
  호출한 곳으로 값을 반환하려면 return 문장을 쓰고 그 다음에 반환하고자
  하는 수식 입력
\item
  함수 안에서 return 뒤에 아무것도 쓰지 않으면 값이 반환되지 않고 호출된
  곳으로 복귀\texttt{return:}
\item
  함수도 변수 종류 중 하나
  \texttt{type(\textless{}function\textgreater{})} -\textgreater{}
  function \#\#\#\#\# 함수 선언

\begin{verbatim}
def <function_name>(<parameter>):
code
return
\end{verbatim}
\item
  parameter는 함수를 호출할 때 전달하는 데이터를 함수 내에서 변수로 사용
  \#\#\#\#\# 함수 호출

\begin{verbatim}
<function_name>(<argument>)
\end{verbatim}
\item
  argument는 함수를 호출할 때 데이터를 전달
\end{itemize}

    \begin{Verbatim}[commandchars=\\\{\}]
{\color{incolor}In [{\color{incolor}3}]:} \PY{c+c1}{\PYZsh{} 소수찾기 함수}
        \PY{k}{def} \PY{n+nf}{is\PYZus{}prime}\PY{p}{(}\PY{n}{num}\PY{p}{)}\PY{p}{:}
            \PY{k}{for} \PY{n}{i} \PY{o+ow}{in} \PY{n+nb}{range}\PY{p}{(}\PY{l+m+mi}{2}\PY{p}{,} \PY{n}{num}\PY{p}{)}\PY{p}{:}
                \PY{n}{num}\PY{o}{\PYZpc{}}\PY{k}{i} == 0       \PYZsh{}n을 i로 나눠서 나머지가 0이면 n은 소수가 아니므로 False 반환하고 함수 종류
                \PY{k}{return} \PY{k+kc}{False}    
            \PY{k}{return} \PY{k+kc}{True}         \PY{c+c1}{\PYZsh{}반복을 종료했는데 함수가 종료되지 않았다면 n의 약수는 없는 것이므로 n은 소수, True반환}
        \PY{n}{is\PYZus{}prime}\PY{p}{(}\PY{l+m+mi}{69}\PY{p}{)}
\end{Verbatim}


\begin{Verbatim}[commandchars=\\\{\}]
{\color{outcolor}Out[{\color{outcolor}3}]:} False
\end{Verbatim}
            
    \begin{Verbatim}[commandchars=\\\{\}]
{\color{incolor}In [{\color{incolor}18}]:} \PY{k}{def} \PY{n+nf}{calc\PYZus{}func}\PY{p}{(}\PY{n}{num1}\PY{p}{,} \PY{n}{num2}\PY{p}{)}\PY{p}{:}                                     
             \PY{k}{return} \PY{n}{num1} \PY{o}{+} \PY{n}{num2}\PY{p}{,} \PY{n}{num1} \PY{o}{\PYZhy{}} \PY{n}{num2}    \PY{c+c1}{\PYZsh{}튜플로 결과 출력 가능}
         
         \PY{n}{tu} \PY{o}{=} \PY{n}{calc\PYZus{}func}\PY{p}{(}\PY{l+m+mi}{5}\PY{p}{,} \PY{l+m+mi}{3}\PY{p}{)}
         \PY{n+nb}{print}\PY{p}{(}\PY{n}{tu}\PY{p}{)}
\end{Verbatim}


    \begin{Verbatim}[commandchars=\\\{\}]
(8, 2)

    \end{Verbatim}

    \subparagraph{keyword argument}\label{keyword-argument}

\begin{itemize}
\tightlist
\item
  positional argument로 argument를 넣지 않고 변수이름을 지정해 값을 대입
\end{itemize}

\subparagraph{default parameter}\label{default-parameter}

\begin{itemize}
\tightlist
\item
  함수를 호출하고 파라미터에 대한 아규먼트가 없을 때 디폴트로 설정된
  값이 파라미터 변수로 사용
\end{itemize}

둘다 지정한 값이 맨 뒤로 가야 에러가 나지 않음

    \begin{Verbatim}[commandchars=\\\{\}]
{\color{incolor}In [{\color{incolor}10}]:} \PY{k}{def} \PY{n+nf}{minus}\PY{p}{(}\PY{n}{num1}\PY{p}{,} \PY{n}{num2}\PY{p}{)}\PY{p}{:}
             \PY{n+nb}{print}\PY{p}{(}\PY{n}{num1}\PY{o}{\PYZhy{}}\PY{n}{num2}\PY{p}{)}
         \PY{n}{minus}\PY{p}{(}\PY{n}{num2}\PY{o}{=}\PY{l+m+mi}{3}\PY{p}{,} \PY{n}{num1}\PY{o}{=}\PY{l+m+mi}{8}\PY{p}{)}         \PY{c+c1}{\PYZsh{}minus(num=8, 3), minus(3, num1=8)은 에러발생}
\end{Verbatim}


    \begin{Verbatim}[commandchars=\\\{\}]
5

    \end{Verbatim}

    \#\#\#\# *args,*\emph{kwargs \#\#\#\#\# }args - argument의 개수가
가변적일 때, 함수의 파라미터 영역에 사용 - tuple 데이터 타입으로 입력

    \begin{Verbatim}[commandchars=\\\{\}]
{\color{incolor}In [{\color{incolor}32}]:} \PY{k}{def} \PY{n+nf}{dup}\PY{p}{(}\PY{o}{*}\PY{n}{args}\PY{p}{)}\PY{p}{:}
             \PY{n}{result} \PY{o}{=} \PY{l+m+mi}{1}
             \PY{k}{for} \PY{n}{i} \PY{o+ow}{in} \PY{n+nb}{range}\PY{p}{(}\PY{l+m+mi}{1}\PY{p}{,} \PY{n+nb}{len}\PY{p}{(}\PY{n}{args}\PY{p}{)}\PY{o}{+}\PY{l+m+mi}{1}\PY{p}{)}\PY{p}{:}  \PY{c+c1}{\PYZsh{}range(1, args)로 하면 \PYZsq{}tuple\PYZsq{} object cannot be interpreted as an integer에러}
                 \PY{n}{result} \PY{o}{=} \PY{n}{result} \PY{o}{*} \PY{n}{i}
             \PY{k}{return} \PY{n}{result}
         
         \PY{n}{dup}\PY{p}{(}\PY{l+m+mi}{1}\PY{p}{,} \PY{l+m+mi}{2}\PY{p}{,} \PY{l+m+mi}{3}\PY{p}{,} \PY{l+m+mi}{4}\PY{p}{,} \PY{l+m+mi}{5}\PY{p}{,} \PY{l+m+mi}{6}\PY{p}{,} \PY{l+m+mi}{7}\PY{p}{,} \PY{l+m+mi}{8}\PY{p}{,} \PY{l+m+mi}{9}\PY{p}{,} \PY{l+m+mi}{10}\PY{p}{)}
\end{Verbatim}


\begin{Verbatim}[commandchars=\\\{\}]
{\color{outcolor}Out[{\color{outcolor}32}]:} 3628800
\end{Verbatim}
            
    \begin{Verbatim}[commandchars=\\\{\}]
{\color{incolor}In [{\color{incolor}39}]:} \PY{k}{def} \PY{n+nf}{avg\PYZus{}func}\PY{p}{(}\PY{o}{*}\PY{n}{args}\PY{p}{)}\PY{p}{:}
             \PY{k}{return} \PY{n+nb}{sum}\PY{p}{(}\PY{n}{args}\PY{p}{)}\PY{o}{/}\PY{n+nb}{len}\PY{p}{(}\PY{n}{args}\PY{p}{)}
         \PY{n}{tu} \PY{o}{=} \PY{p}{(}\PY{l+m+mi}{1}\PY{p}{,} \PY{l+m+mi}{2}\PY{p}{,} \PY{l+m+mi}{3}\PY{p}{,} \PY{l+m+mi}{4}\PY{p}{,} \PY{l+m+mi}{5}\PY{p}{,} \PY{l+m+mi}{6}\PY{p}{,} \PY{l+m+mi}{7}\PY{p}{,} \PY{l+m+mi}{8}\PY{p}{,} \PY{l+m+mi}{9}\PY{p}{,} \PY{l+m+mi}{10}\PY{p}{)}    
         \PY{n}{avg\PYZus{}func}\PY{p}{(}\PY{o}{*}\PY{n}{tu}\PY{p}{)}
\end{Verbatim}


\begin{Verbatim}[commandchars=\\\{\}]
{\color{outcolor}Out[{\color{outcolor}39}]:} 5.5
\end{Verbatim}
            
    \begin{Verbatim}[commandchars=\\\{\}]
{\color{incolor}In [{\color{incolor}89}]:} \PY{c+c1}{\PYZsh{}예시}
         \PY{k}{def} \PY{n+nf}{test}\PY{p}{(}\PY{o}{*}\PY{n}{args}\PY{p}{)}\PY{p}{:}
             \PY{k}{return} \PY{n+nb}{print}\PY{p}{(}\PY{n}{args}\PY{p}{)}
             
         \PY{n}{ls} \PY{o}{=} \PY{p}{[}\PY{l+m+mi}{1}\PY{p}{,} \PY{l+m+mi}{2}\PY{p}{,} \PY{l+m+mi}{3}\PY{p}{,} \PY{l+m+mi}{4}\PY{p}{,} \PY{l+m+mi}{5}\PY{p}{]}
         \PY{n}{test}\PY{p}{(}\PY{o}{*}\PY{n}{ls}\PY{p}{)}
         \PY{n}{test}\PY{p}{(}\PY{n}{ls}\PY{p}{)} \PY{c+c1}{\PYZsh{}리스트는 바로 하면 하나의 리스트로 입력}
\end{Verbatim}


    \begin{Verbatim}[commandchars=\\\{\}]
(1, 2, 3, 4, 5)
([1, 2, 3, 4, 5],)

    \end{Verbatim}

    \subparagraph{2.**kwargs - 함수의 파라미터 영역에 사용, 가변적인 키워드
argument 사용 시 - 파라미터로 받는 데이터는 dictionary 데이터
타입}\label{kwargs---uxd568uxc218uxc758-uxd30cuxb77cuxbbf8uxd130-uxc601uxc5eduxc5d0-uxc0acuxc6a9-uxac00uxbcc0uxc801uxc778-uxd0a4uxc6ccuxb4dc-argument-uxc0acuxc6a9-uxc2dc---uxd30cuxb77cuxbbf8uxd130uxb85c-uxbc1buxb294-uxb370uxc774uxd130uxb294-dictionary-uxb370uxc774uxd130-uxd0c0uxc785}

    \begin{Verbatim}[commandchars=\\\{\}]
{\color{incolor}In [{\color{incolor}43}]:} \PY{c+c1}{\PYZsh{}예시}
         \PY{k}{def} \PY{n+nf}{avg\PYZus{}func}\PY{p}{(}\PY{o}{*}\PY{o}{*}\PY{n}{kwargs}\PY{p}{)}\PY{p}{:}
             \PY{n+nb}{print}\PY{p}{(}\PY{n}{kwargs}\PY{p}{)}
             \PY{n+nb}{print}\PY{p}{(}\PY{n+nb}{type}\PY{p}{(}\PY{n}{kwargs}\PY{p}{)}\PY{p}{)}
             \PY{n}{total}\PY{p}{,} \PY{n}{count} \PY{o}{=} \PY{l+m+mi}{0}\PY{p}{,} \PY{l+m+mi}{0}
             \PY{k}{for} \PY{n}{subject}\PY{p}{,} \PY{n}{point} \PY{o+ow}{in} \PY{n}{kwargs}\PY{o}{.}\PY{n}{items}\PY{p}{(}\PY{p}{)}\PY{p}{:}
                 \PY{n+nb}{print}\PY{p}{(}\PY{n}{subject}\PY{p}{,} \PY{n}{point}\PY{p}{)}
                 \PY{n}{total} \PY{o}{+}\PY{o}{=} \PY{n}{point}
                 \PY{n}{count} \PY{o}{+}\PY{o}{=} \PY{l+m+mi}{1}
             \PY{k}{return} \PY{n}{total}\PY{o}{/}\PY{n}{count}
         
         \PY{n}{avg\PYZus{}func}\PY{p}{(}\PY{n}{korean}\PY{o}{=}\PY{l+m+mi}{100}\PY{p}{,} \PY{n}{english}\PY{o}{=}\PY{l+m+mi}{70}\PY{p}{,} \PY{n}{math}\PY{o}{=}\PY{l+m+mi}{80}\PY{p}{)}
\end{Verbatim}


    \begin{Verbatim}[commandchars=\\\{\}]
\{'korean': 100, 'english': 70, 'math': 80\}
<class 'dict'>
korean 100
english 70
math 80

    \end{Verbatim}

\begin{Verbatim}[commandchars=\\\{\}]
{\color{outcolor}Out[{\color{outcolor}43}]:} 83.33333333333333
\end{Verbatim}
            
    \begin{Verbatim}[commandchars=\\\{\}]
{\color{incolor}In [{\color{incolor}83}]:} \PY{k}{def} \PY{n+nf}{a}\PY{p}{(}\PY{o}{*}\PY{o}{*}\PY{n}{kwargs}\PY{p}{)}\PY{p}{:}
             \PY{n}{sum\PYZus{}a} \PY{o}{=} \PY{l+m+mi}{0}
             \PY{n}{lgth} \PY{o}{=} \PY{l+m+mi}{0}
             \PY{k}{for} \PY{n}{score} \PY{o+ow}{in} \PY{n}{kwargs}\PY{o}{.}\PY{n}{values}\PY{p}{(}\PY{p}{)}\PY{p}{:} \PY{c+c1}{\PYZsh{}kwargs.value()도 가능}
                 \PY{n}{sum\PYZus{}a} \PY{o}{+}\PY{o}{=} \PY{n}{score}
                 \PY{n+nb}{print}\PY{p}{(}\PY{n}{sum\PYZus{}a}\PY{p}{)}
             \PY{k}{return} \PY{n}{sum\PYZus{}a}
             
         \PY{n}{a}\PY{p}{(}\PY{n}{kim} \PY{o}{=} \PY{l+m+mi}{90}\PY{p}{,} \PY{n}{park} \PY{o}{=} \PY{l+m+mi}{50}\PY{p}{,} \PY{n}{sun} \PY{o}{=} \PY{l+m+mi}{90}\PY{p}{)}
\end{Verbatim}


    \begin{Verbatim}[commandchars=\\\{\}]
90
140
230

    \end{Verbatim}

\begin{Verbatim}[commandchars=\\\{\}]
{\color{outcolor}Out[{\color{outcolor}83}]:} 230
\end{Verbatim}
            
    \begin{Verbatim}[commandchars=\\\{\}]
{\color{incolor}In [{\color{incolor}85}]:} \PY{k}{def} \PY{n+nf}{a}\PY{p}{(}\PY{o}{*}\PY{o}{*}\PY{n}{kwargs}\PY{p}{)}\PY{p}{:}
             \PY{n}{sum\PYZus{}a} \PY{o}{=} \PY{l+m+mi}{0}
             \PY{n}{lgth} \PY{o}{=} \PY{l+m+mi}{0}
             \PY{k}{for} \PY{n}{name}\PY{p}{,} \PY{n}{score} \PY{o+ow}{in} \PY{n}{kwargs}\PY{o}{.}\PY{n}{items}\PY{p}{(}\PY{p}{)}\PY{p}{:} 
                 \PY{n}{sum\PYZus{}a} \PY{o}{+}\PY{o}{=} \PY{n}{score}
                 \PY{n+nb}{print}\PY{p}{(}\PY{n}{sum\PYZus{}a}\PY{p}{)}
             \PY{k}{return} \PY{n}{sum\PYZus{}a}\PY{o}{/}\PY{n+nb}{len}\PY{p}{(}\PY{n}{name}\PY{p}{)}  \PY{c+c1}{\PYZsh{}len()에는 str만}
             
         \PY{n+nb}{round}\PY{p}{(}\PY{n}{a}\PY{p}{(}\PY{n}{kim} \PY{o}{=} \PY{l+m+mi}{90}\PY{p}{,} \PY{n}{park} \PY{o}{=} \PY{l+m+mi}{50}\PY{p}{,} \PY{n}{sun} \PY{o}{=} \PY{l+m+mi}{90}\PY{p}{)}\PY{p}{,} \PY{l+m+mi}{2}\PY{p}{)}
\end{Verbatim}


    \begin{Verbatim}[commandchars=\\\{\}]
90
140
230

    \end{Verbatim}

\begin{Verbatim}[commandchars=\\\{\}]
{\color{outcolor}Out[{\color{outcolor}85}]:} 76.67
\end{Verbatim}
            
    \subparagraph{3.*args, **kwargs 함께 사용 - 키워드가 없는 argument들은
args로 받아들여지고, 키워드가 있는 argument들은 kwargs로
받아짐}\label{args-kwargs-uxd568uxaed8-uxc0acuxc6a9---uxd0a4uxc6ccuxb4dcuxac00-uxc5c6uxb294-argumentuxb4e4uxc740-argsuxb85c-uxbc1buxc544uxb4e4uxc5ecuxc9c0uxace0-uxd0a4uxc6ccuxb4dcuxac00-uxc788uxb294-argumentuxb4e4uxc740-kwargsuxb85c-uxbc1buxc544uxc9d0}

    \begin{Verbatim}[commandchars=\\\{\}]
{\color{incolor}In [{\color{incolor}46}]:} \PY{c+c1}{\PYZsh{}예시}
         \PY{k}{def} \PY{n+nf}{test\PYZus{}func}\PY{p}{(}\PY{o}{*}\PY{n}{args}\PY{p}{,} \PY{o}{*}\PY{o}{*}\PY{n}{kwargs}\PY{p}{)}\PY{p}{:}
             \PY{n+nb}{print}\PY{p}{(}\PY{n}{args}\PY{p}{)}
             \PY{n+nb}{print}\PY{p}{(}\PY{n}{kwargs}\PY{p}{)}
         \PY{n}{test\PYZus{}func}\PY{p}{(}\PY{l+m+mi}{1}\PY{p}{,} \PY{l+m+mi}{2}\PY{p}{,} \PY{l+m+mi}{3}\PY{p}{,} \PY{l+s+s2}{\PYZdq{}}\PY{l+s+s2}{fastcampus}\PY{l+s+s2}{\PYZdq{}}\PY{p}{,} \PY{l+s+s2}{\PYZdq{}}\PY{l+s+s2}{datascience}\PY{l+s+s2}{\PYZdq{}}\PY{p}{,} \PY{n}{korean} \PY{o}{=} \PY{l+m+mi}{100}\PY{p}{,} \PY{n}{english} \PY{o}{=} \PY{l+m+mi}{70}\PY{p}{,} \PY{n}{math} \PY{o}{=} \PY{l+m+mi}{80}\PY{p}{)}
\end{Verbatim}


    \begin{Verbatim}[commandchars=\\\{\}]
(1, 2, 3, 'fastcampus', 'datascience')
\{'korean': 100, 'english': 70, 'math': 80\}

    \end{Verbatim}

    \paragraph{docstring}\label{docstring}

\begin{verbatim}
  - pep 20 : Zen of Python 참조
  - 함수에 대한 설명, 함수를 선언한 코드 아래에 문자열로 입력
\end{verbatim}

    \begin{Verbatim}[commandchars=\\\{\}]
{\color{incolor}In [{\color{incolor}94}]:} \PY{k}{def} \PY{n+nf}{str\PYZus{}print}\PY{p}{(}\PY{n}{string}\PY{p}{)}\PY{p}{:}
             \PY{l+s+sd}{\PYZdq{}\PYZdq{}\PYZdq{}}
         \PY{l+s+sd}{    str\PYZus{}print returns string word.}
         \PY{l+s+sd}{    \PYZdq{}\PYZdq{}\PYZdq{}}
             \PY{k}{return} \PY{l+s+s2}{\PYZdq{}}\PY{l+s+s2}{input word : }\PY{l+s+si}{\PYZob{}\PYZcb{}}\PY{l+s+s2}{\PYZdq{}}\PY{o}{.}\PY{n}{format}\PY{p}{(}\PY{n}{string}\PY{p}{)}
         
         \PY{n}{str\PYZus{}print}\PY{p}{(}\PY{l+s+s2}{\PYZdq{}}\PY{l+s+s2}{fast}\PY{l+s+s2}{\PYZdq{}}\PY{p}{)}  \PY{c+c1}{\PYZsh{}문자 argument는 꼭 \PYZdq{}\PYZdq{}표시}
\end{Verbatim}


\begin{Verbatim}[commandchars=\\\{\}]
{\color{outcolor}Out[{\color{outcolor}94}]:} 'input word : fast'
\end{Verbatim}
            
    \begin{Verbatim}[commandchars=\\\{\}]
{\color{incolor}In [{\color{incolor}99}]:} \PY{n}{help}\PY{p}{(}\PY{n}{str\PYZus{}print}\PY{p}{)}
         \PY{n+nb}{print}\PY{p}{(}\PY{n}{str\PYZus{}print}\PY{o}{.}\PY{n+nv+vm}{\PYZus{}\PYZus{}doc\PYZus{}\PYZus{}}\PY{p}{)}
\end{Verbatim}


    \begin{Verbatim}[commandchars=\\\{\}]
Help on function str\_print in module \_\_main\_\_:

str\_print(string)
    str\_print returns string word.


    str\_print returns string word.
    

    \end{Verbatim}

    \paragraph{scope}\label{scope}

\begin{itemize}
\tightlist
\item
  변수나 함수를 선언할 때 해당 함수는 변수를 사용할 수 있는 영역 스코프
  존재
\item
  global, local
\end{itemize}

    \subparagraph{1. global}\label{global}

\begin{verbatim}
- 전역(전역함수, 전역변수)
- 메모리에 global, local영역이 있어서 함수를 선언하면 함수(함수는 global영역에 선언)에 대한 local영역이
  생성되고 똑같은 변수라도 그 안에서만 선언
- global() : 현재까지 실행한 global영역의 값
\end{verbatim}

    \begin{Verbatim}[commandchars=\\\{\}]
{\color{incolor}In [{\color{incolor}115}]:} \PY{c+c1}{\PYZsh{}예시}
          \PY{n}{gv1}\PY{p}{,} \PY{n}{gv2} \PY{o}{=} \PY{l+m+mi}{1}\PY{p}{,} \PY{l+m+mi}{2}
          \PY{k}{def} \PY{n+nf}{print\PYZus{}variable}\PY{p}{(}\PY{p}{)}\PY{p}{:}
              \PY{n}{gv1}\PY{p}{,} \PY{n}{gv2} \PY{o}{=} \PY{l+m+mi}{10}\PY{p}{,} \PY{l+m+mi}{20}
              \PY{k}{return} \PY{n}{gv1}\PY{p}{,} \PY{n}{gv2}
          \PY{n}{gv1}\PY{p}{,} \PY{n}{gv2}
\end{Verbatim}


\begin{Verbatim}[commandchars=\\\{\}]
{\color{outcolor}Out[{\color{outcolor}115}]:} (1, 2)
\end{Verbatim}
            
    \begin{Verbatim}[commandchars=\\\{\}]
{\color{incolor}In [{\color{incolor}109}]:} \PY{n}{print\PYZus{}value}\PY{p}{(}\PY{p}{)}
\end{Verbatim}


\begin{Verbatim}[commandchars=\\\{\}]
{\color{outcolor}Out[{\color{outcolor}109}]:} (10, 20)
\end{Verbatim}
            
    \begin{Verbatim}[commandchars=\\\{\}]
{\color{incolor}In [{\color{incolor}6}]:} \PY{n}{gv1} \PY{o}{=} \PY{l+m+mi}{1}
        \PY{k}{def} \PY{n+nf}{change\PYZus{}gv}\PY{p}{(}\PY{p}{)}\PY{p}{:}
            \PY{k}{global} \PY{n}{gv1}   \PY{c+c1}{\PYZsh{}global로 선언하기 전에 local영역에서 동일한 변수를 선언하면 error}
            \PY{n}{gv1} \PY{o}{=} \PY{l+m+mi}{100}
            \PY{n+nb}{print}\PY{p}{(}\PY{n}{gv1}\PY{p}{)}
        \PY{n}{change\PYZus{}gv}\PY{p}{(}\PY{p}{)}      \PY{c+c1}{\PYZsh{}함수를 실행하고 변수를 입력해야 전역변수 변경이 됨}
        \PY{n}{gv1} 
\end{Verbatim}


    \begin{Verbatim}[commandchars=\\\{\}]
100

    \end{Verbatim}

\begin{Verbatim}[commandchars=\\\{\}]
{\color{outcolor}Out[{\color{outcolor}6}]:} 100
\end{Verbatim}
            
    \begin{Verbatim}[commandchars=\\\{\}]
{\color{incolor}In [{\color{incolor}2}]:} \PY{n}{change\PYZus{}gv}\PY{p}{(}\PY{p}{)}
\end{Verbatim}


    \begin{Verbatim}[commandchars=\\\{\}]
100

    \end{Verbatim}

    \paragraph{2.local}\label{local}

\begin{itemize}
\tightlist
\item
  지역(지역함수, 지역변수)
\item
  locals() : local변수
\item
  print(locals(){[}""{]})
\end{itemize}

    \begin{Verbatim}[commandchars=\\\{\}]
{\color{incolor}In [{\color{incolor}149}]:} \PY{c+c1}{\PYZsh{}????????}
          \PY{n}{gv3}\PY{p}{,} \PY{n}{gv4} \PY{o}{=} \PY{l+m+mi}{3}\PY{p}{,} \PY{l+m+mi}{4}
          \PY{k}{def} \PY{n+nf}{print\PYZus{}locals}\PY{p}{(}\PY{p}{)}\PY{p}{:}
              \PY{n}{gv3} \PY{o}{=} \PY{l+m+mi}{30}
              \PY{n}{gv4} \PY{o}{=} \PY{l+m+mi}{40}
              \PY{k}{def} \PY{n+nf}{plus}\PY{p}{(}\PY{n}{a}\PY{p}{,} \PY{n}{b}\PY{p}{)}\PY{p}{:}
                  \PY{k}{return} \PY{n}{a}\PY{o}{+}\PY{n}{b}
          
          \PY{n}{print\PYZus{}locals}\PY{p}{(}\PY{p}{)}
\end{Verbatim}


    \subparagraph{3.local영역에서 global변수의
변경}\label{localuxc601uxc5eduxc5d0uxc11c-globaluxbcc0uxc218uxc758-uxbcc0uxacbd}

    \begin{Verbatim}[commandchars=\\\{\}]
{\color{incolor}In [{\color{incolor}160}]:} \PY{c+c1}{\PYZsh{}\PYZsh{}\PYZsh{}\PYZsh{}\PYZsh{}예시}
          \PY{n}{gv} \PY{o}{=} \PY{l+m+mi}{10}
          \PY{k}{def} \PY{n+nf}{change\PYZus{}gv}\PY{p}{(}\PY{n}{data}\PY{p}{)}\PY{p}{:}
              \PY{n+nb}{print}\PY{p}{(}\PY{l+s+s2}{\PYZdq{}}\PY{l+s+s2}{change\PYZus{}gv}\PY{l+s+s2}{\PYZdq{}}\PY{p}{)}   
              \PY{n}{gv} \PY{o}{=} \PY{n}{data}   
              \PY{n+nb}{print}\PY{p}{(}\PY{n+nb}{locals}\PY{p}{(}\PY{p}{)}\PY{p}{)}         \PY{c+c1}{\PYZsh{}global gv가 없을 땐 gv가 지역변수로 생성}
              \PY{n+nb}{print}\PY{p}{(}\PY{l+s+s2}{\PYZdq{}}\PY{l+s+s2}{global gv : }\PY{l+s+si}{\PYZob{}\PYZcb{}}\PY{l+s+s2}{\PYZdq{}}\PY{o}{.}\PY{n}{format}\PY{p}{(}\PY{n+nb}{globals}\PY{p}{(}\PY{p}{)}\PY{p}{[}\PY{l+s+s2}{\PYZdq{}}\PY{l+s+s2}{gv}\PY{l+s+s2}{\PYZdq{}}\PY{p}{]}\PY{p}{)}\PY{p}{)}
          
          \PY{n+nb}{print}\PY{p}{(}\PY{n}{gv}\PY{p}{)}
          \PY{n}{change\PYZus{}gv}\PY{p}{(}\PY{l+m+mi}{100}\PY{p}{)}
\end{Verbatim}


    \begin{Verbatim}[commandchars=\\\{\}]
10
change\_gv
\{'gv': 100, 'data': 100\}
global gv : 10

    \end{Verbatim}

    \begin{Verbatim}[commandchars=\\\{\}]
{\color{incolor}In [{\color{incolor}167}]:} \PY{c+c1}{\PYZsh{}\PYZsh{}\PYZsh{}\PYZsh{}\PYZsh{}예시}
          \PY{n}{gv} \PY{o}{=} \PY{l+m+mi}{10}
          \PY{k}{def} \PY{n+nf}{change\PYZus{}gv}\PY{p}{(}\PY{n}{data}\PY{p}{)}\PY{p}{:}
              \PY{n+nb}{print}\PY{p}{(}\PY{l+s+s2}{\PYZdq{}}\PY{l+s+s2}{change\PYZus{}gv}\PY{l+s+s2}{\PYZdq{}}\PY{p}{)}
              \PY{k}{global} \PY{n}{gv}                   
              \PY{n}{gv} \PY{o}{=} \PY{n}{data}
              \PY{n+nb}{print}\PY{p}{(}\PY{n+nb}{locals}\PY{p}{(}\PY{p}{)}\PY{p}{)}
              \PY{n+nb}{print}\PY{p}{(}\PY{l+s+s2}{\PYZdq{}}\PY{l+s+s2}{global gv : }\PY{l+s+si}{\PYZob{}\PYZcb{}}\PY{l+s+s2}{\PYZdq{}}\PY{o}{.}\PY{n}{format}\PY{p}{(}\PY{n+nb}{globals}\PY{p}{(}\PY{p}{)}\PY{p}{[}\PY{l+s+s2}{\PYZdq{}}\PY{l+s+s2}{gv}\PY{l+s+s2}{\PYZdq{}}\PY{p}{]}\PY{p}{)}\PY{p}{)}
          
          \PY{n+nb}{print}\PY{p}{(}\PY{n}{gv}\PY{p}{)}
          \PY{n}{change\PYZus{}gv}\PY{p}{(}\PY{l+m+mi}{100}\PY{p}{)}
          \PY{n}{gv}
\end{Verbatim}


    \begin{Verbatim}[commandchars=\\\{\}]
10
change\_gv
\{'data': 100\}
global gv : 100

    \end{Verbatim}

\begin{Verbatim}[commandchars=\\\{\}]
{\color{outcolor}Out[{\color{outcolor}167}]:} 100
\end{Verbatim}
            
    \paragraph{inner function}\label{inner-function}

\begin{itemize}
\tightlist
\item
  코드를 숨기거나, 수정할 수 없도록 만들 때
\end{itemize}

    \begin{Verbatim}[commandchars=\\\{\}]
{\color{incolor}In [{\color{incolor}169}]:} \PY{c+c1}{\PYZsh{}예시}
          \PY{k}{def} \PY{n+nf}{outer}\PY{p}{(}\PY{n}{a}\PY{p}{,} \PY{n}{b}\PY{p}{)}\PY{p}{:}
              \PY{k}{def} \PY{n+nf}{inner}\PY{p}{(}\PY{n}{c}\PY{p}{,} \PY{n}{d}\PY{p}{)}\PY{p}{:}
                  \PY{k}{return} \PY{n}{c}\PY{o}{+}\PY{n}{d}
              \PY{k}{return} \PY{n}{inner}\PY{p}{(}\PY{n}{a}\PY{p}{,} \PY{n}{b}\PY{p}{)}
          
          \PY{n}{outer}\PY{p}{(}\PY{l+m+mi}{4}\PY{p}{,} \PY{l+m+mi}{7}\PY{p}{)}
\end{Verbatim}


\begin{Verbatim}[commandchars=\\\{\}]
{\color{outcolor}Out[{\color{outcolor}169}]:} 11
\end{Verbatim}
            
    \paragraph{lambda function}\label{lambda-function}

\begin{itemize}
\tightlist
\item
  간단한 함수를 선언하지 않고 함수처럼 사용
\item
  lambda
  \texttt{\textless{}parameters\textgreater{}\ :\ \textless{}return\_value\textgreater{}}
\end{itemize}

    \begin{Verbatim}[commandchars=\\\{\}]
{\color{incolor}In [{\color{incolor}1}]:} \PY{c+c1}{\PYZsh{}예시}
        \PY{k}{def} \PY{n+nf}{sum\PYZus{}func}\PY{p}{(}\PY{n}{x}\PY{p}{,} \PY{n}{y}\PY{p}{)}\PY{p}{:}
            \PY{k}{return} \PY{n}{x}\PY{o}{+}\PY{n}{y}
        \PY{n}{sum\PYZus{}func}\PY{p}{(}\PY{l+m+mi}{5}\PY{p}{,} \PY{l+m+mi}{6}\PY{p}{)}
\end{Verbatim}


\begin{Verbatim}[commandchars=\\\{\}]
{\color{outcolor}Out[{\color{outcolor}1}]:} 11
\end{Verbatim}
            
    \begin{Verbatim}[commandchars=\\\{\}]
{\color{incolor}In [{\color{incolor}3}]:} \PY{n}{sum\PYZus{}func2} \PY{o}{=} \PY{k}{lambda} \PY{n}{x}\PY{p}{,} \PY{n}{y} \PY{p}{:} \PY{n}{x}\PY{o}{+}\PY{n}{y}
        \PY{n}{sum\PYZus{}func2}\PY{p}{(}\PY{l+m+mi}{5}\PY{p}{,} \PY{l+m+mi}{6}\PY{p}{)}
\end{Verbatim}


\begin{Verbatim}[commandchars=\\\{\}]
{\color{outcolor}Out[{\color{outcolor}3}]:} 11
\end{Verbatim}
            
    \begin{Verbatim}[commandchars=\\\{\}]
{\color{incolor}In [{\color{incolor}6}]:} \PY{n}{full\PYZus{}name} \PY{o}{=} \PY{k}{lambda} \PY{n}{fn}\PY{p}{,} \PY{n}{ln}\PY{p}{:} \PY{n}{fn}\PY{o}{+}\PY{l+s+s2}{\PYZdq{}}\PY{l+s+s2}{ }\PY{l+s+s2}{\PYZdq{}}\PY{o}{+} \PY{n}{ln}
        \PY{n}{full\PYZus{}name}\PY{p}{(}\PY{l+s+s2}{\PYZdq{}}\PY{l+s+s2}{John}\PY{l+s+s2}{\PYZdq{}}\PY{p}{,} \PY{l+s+s2}{\PYZdq{}}\PY{l+s+s2}{Smith}\PY{l+s+s2}{\PYZdq{}}\PY{p}{)}
\end{Verbatim}


\begin{Verbatim}[commandchars=\\\{\}]
{\color{outcolor}Out[{\color{outcolor}6}]:} 'John Smith'
\end{Verbatim}
            
    \begin{Verbatim}[commandchars=\\\{\}]
{\color{incolor}In [{\color{incolor}8}]:} \PY{c+c1}{\PYZsh{} 디폴트 파라미터}
        \PY{n}{animal} \PY{o}{=} \PY{k}{lambda} \PY{n}{a}\PY{o}{=}\PY{l+s+s2}{\PYZdq{}}\PY{l+s+s2}{dog}\PY{l+s+s2}{\PYZdq{}}\PY{p}{,} \PY{n}{b}\PY{o}{=}\PY{l+s+s2}{\PYZdq{}}\PY{l+s+s2}{cat}\PY{l+s+s2}{\PYZdq{}} \PY{p}{:} \PY{n}{a} \PY{o}{+}\PY{l+s+s2}{\PYZdq{}}\PY{l+s+s2}{, }\PY{l+s+s2}{\PYZdq{}}\PY{o}{+} \PY{n}{b}
        \PY{n}{animal}\PY{p}{(}\PY{l+s+s2}{\PYZdq{}}\PY{l+s+s2}{pig}\PY{l+s+s2}{\PYZdq{}}\PY{p}{)}\PY{p}{,} \PY{n}{animal}\PY{p}{(}\PY{p}{)}
\end{Verbatim}


\begin{Verbatim}[commandchars=\\\{\}]
{\color{outcolor}Out[{\color{outcolor}8}]:} ('pig, cat', 'dog, cat')
\end{Verbatim}
            
    \begin{Verbatim}[commandchars=\\\{\}]
{\color{incolor}In [{\color{incolor} }]:} \PY{c+c1}{\PYZsh{} 람다함수의 파라미터 응용}
\end{Verbatim}



    % Add a bibliography block to the postdoc
    
    
    
    \end{document}
